\documentclass{guide}

\title{RFID kaartlezer via GPIO}
\author{Bart van Eijkelenburg}
\level{2}

\begin{document}
\section{Aansluiten}
Bedrading volgens schema~\ref{tab:pins}\footnote{Bron: \url{https://github.com/ondryaso/pi-rc522}}. Let op, alle 8 pinnen aansluiten!

\begin{table}[h]
  \centering
  \begin{tabular}{|r|c|c|l|}
    \hline
    \bf Board pin name & \bf Board pin & \bf Physical RPi pin & \bf RPi pin name \\
    \hline
      \tt SDA & 1 & 24 & \tt GPIO8, CE0 \\
    \hline
      \tt SCK & 2 & 23 & \tt GPIO11, SCKL \\
    \hline
      \tt MOSI & 3 & 19 & \tt GPIO10, MOSI \\
    \hline
      \tt MISO & 4 & 21 & \tt GPIO9, MISO \\
    \hline
      \tt IRQ & 5 & 18 & \tt GPIO24 \\
    \hline
      \tt GND & 6 & 6, 9, 20, 25 & \tt GND \\
    \hline
      \tt RST & 7 & 22 & \tt GPIO25 \\
    \hline
      \tt 3.3V & 8 & 1,17 & \tt 3V3 \\
    \hline
  \end{tabular}
  \caption{Aansluiten RC522 op Raspberry Pi}\label{tab:pins}
\end{table}

\section{Raspberry Pi instellen}
Volg onderstaande stappen:

  \begin{enumerate}
    \item \texttt{raspi-config} $\to$ Interfacing options $\to$ SPI inschakelen
    \item \texttt{sudo reboot}
    \item Installeer de SPI-Py library:
      \begin{bash}
git clone https://github.com/lthiery/SPI-Py
cd SPI-Py
sudo python3 setup.py install
      \end{bash}

    \item Installeer de pi-rc522 bestanden:
      \begin{bash}
git clone https://github.com/ondryaso/pi-rc522.git
cd pi-rc522
sudo python3 setup.py install
      \end{bash}
  \end{enumerate}

\newpage
\section{Voorbeeldcode}
Werkende voorbeeldcode\footnote{Bron: \url{https://github.com/ondryaso/pi-rc522}}:

\begin{python}[caption={De GPIO library inladen}, label=code:PIN_aansturen]
from pirc522 import RFID

rdr = RFID()

try:
    while True:
        rdr.wait_for_tag()
        (error, tag_type) = rdr.request()

        if not error:
            print("Tag detected")
            (error, uid) = rdr.anticoll()

            if not error:
                print("UID: " + str(uid)) # Select Tag is required before Auth

                if not rdr.select_tag(uid): # Auth for block 10 (block 2 of sector 2) using
                                            # default shipping key A
                    if not rdr.card_auth(rdr.auth_a, 10, [0xFF, 0xFF, 0xFF, 0xFF, 0xFF, 0xFF], uid):
                        # This will print something like (False, [0, 0, 0, 0, 0, 0, 0, 0, 0$
                        print("Reading block 10: " + str(rdr.read(10)))

                        # Always stop crypto1 when done working
                        rdr.stop_crypto()

except:
    print('exception')
finally: # Calls GPIO cleanup
    rdr.cleanup()
\end{python}
\end{document}
